\section*{Velocity-Acceleration Plot}
\begin{frame}{Velocity-Acceleration Plot}
    The velocity-acceleration plot is a diagram
    with velocity on the $x$-axis
    and (longitudinal) acceleration on the $y$-axis.
    \begin{center}
        \begin{tikzpicture}
            \begin{axis}[axis lines=center, xmin=0, xmax=55, ymin=-3, ymax=3]
                \pgfplotsset{xlabel={$v_x$}, ylabel={$a_x$}}
            \end{axis}
        \end{tikzpicture}
    \end{center}
    This is known as a \textbf{state space} diagram.
\end{frame}

\begin{frame}
    Lets add some limits to the graph.
    We will begin by drawing a free-body diagram of a car.
    \begin{center}
        \begin{tikzpicture}
            \node (freebody) at (0,0) {%
                \includegraphics[width=0.8\textwidth]{../../res/Free Body Car.png}
            };
            \node[above] (CoG) at (0,0) {$CoG$};
            \node[above] (CoP) at (1,0) {$CoP$};
            \node (TyreF) at (-2.15,-2) {};
            \node (TyreR) at (3.05,-2) {};
            \draw[thick,main,-latex] (CoG) -- ++(0,-2) node[right] {$W$};
            \draw[thick,main,-latex] (CoP) -- ++(0,-2) node[right] {$F_L$};
            \draw[thick,main,-latex] (1,0) -- ++(2,0);
            \node[above] at (2,0) {$F_D$};
            \draw[thick,main,latex-] (3.2,-2.1) -- ++(2,0) node[below] {$F_T$};
            \draw[thick,main,latex-] (TyreF) -- ++(0,-1) node[right] {$R_F$};
            \draw[thick,main,latex-] (TyreR) -- ++(0,-1) node[right] {$R_R$};
        \end{tikzpicture}
    \end{center}
\end{frame}

\begin{frame}
    Consider the forces acting horizontally (the $x$-direction):
    \begin{gather*}
        F_T = \frac{P}{v} \\
        F_D = \frac{1}{2} \rho C_D A v^2
    \end{gather*}
    The acceleration of the vehicle is the sum of these forces:
    \begin{align*}
        a_x &= \frac{1}{m} \sum F_x \\
        &= \frac{1}{m} \left(\frac{P}{v} - \frac{1}{2} \rho C_D A v^2\right) \\
        &= \frac{P}{mv} - \frac{\rho C_D A v^2}{2m}
    \end{align*}
    This is known as the \textbf{power limit} of the vehicle.
\end{frame}

\begin{frame}
    Acceleration is also limited by the grip available from the tyres. \\
    First, we calculate the normal force on the tyres:
    $$N = W + F_L = mg + \frac{1}{2} \rho C_L A v^2$$
    For a tyre with a constant coefficient of friction $\mu$, the maximum grip available is:
    $$F_f = \mu N = \mu \left(mg + \frac{1}{2} \rho C_L A v^2\right)$$
    Therefore, the \textbf{traction-limited} acceleration is:
    $$a_x = \frac{F_f}{m}
    = \frac{\mu}{m} \left(mg + \frac{1}{2} \rho C_L A v^2\right)
    = \mu g + \frac{\mu \rho C_L A}{2m} v^2$$
    Real tyres are \textbf{load sensitive},
    meaning that $\mu$ decreases as $N$ increases.
    This means that adding more downforce has diminishing returns.
\end{frame}

\begin{frame}
    Finally, the top speed of the car is limited
    by the top speed of the motor, $\omega_\text{max}$,
    which can be found on the motor's datasheet
    (try searching for \textit{`Emrax 228 datasheet'}). \\~\\
    This is divided by the final drive ratio
    to find the rotational velocity of the wheels,
    and multiplied by the tyre radius to find the linear velocity of the car.
    $$v_\text{max} = \frac{\omega_\text{max} R_0}{\text{FDR}}$$
\end{frame}

\begin{frame}
    Lets plot these three lines on the velocity-acceleration diagram:
\end{frame}