\section*{Validation}

\begin{frame}{Validation}
    Without validating our simulation,
    we must be careful what conclusions we draw from it.
    \\~\\
    We can observe \textbf{trends}
    (e.g.\ laptime decreases as mass decreases). \\
    We cannot use exact \textbf{numerical} values
    (e.g.\ decreasing mass by 1kg decreases laptime by 0.1s).
    \\~\\
    \textbf{How can we validate our simulation?}
    \begin{itemize}
        \item Compare to real test data from the car.
        This requires testing time and data acquisition.
        \item Compare to test data from other sources.
        Not as strong as using data from our car.
        However, this can be used to show that the lapsim functions correctly.
    \end{itemize}
\end{frame}


\begin{frame}{Correlation}
    What can we do if our model doesn't match experimental data?
    \\~\\
    The most important thing is to understand \textbf{why}.
    Compare various testing data channels to simulation data channels
    to diagnose where the problem is occuring.
    \\~\\
    How do we bring our model in line with the data?
    \begin{itemize}
        \item Fix any errors in the model
        \item Use a more complicated physics model
        \item Add \textbf{correlation coefficients}
    \end{itemize}
    If correlation coefficients are used,
    they should be as simple as possible,
    and their use should be fully justified
    (e.g.\ scaling coefficient applied to tyre grip
    to account for tyre degradation).
\end{frame}